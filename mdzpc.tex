\documentclass[a4paper,12pt]{article}

\usepackage{polski}
\usepackage[utf8]{inputenc}

\usepackage[english]{babel}
\usepackage{blindtext}
\usepackage{microtype}
\usepackage{graphicx}
\usepackage{fancyhdr}
\usepackage{amsmath}
\usepackage{index}
\usepackage{textgreek}
\usepackage{hyperref}
\usepackage{listings}
\usepackage{xcolor}


\newcommand{\stirlingone}[2]{\genfrac{[}{]}{0pt}{}{#1}{#2}}

\definecolor{codegreen}{rgb}{0,0.6,0}
\definecolor{codegray}{rgb}{0.5,0.5,0.5}
\definecolor{codepurple}{rgb}{0.58,0,0.82}
\definecolor{backcolour}{rgb}{0.95,0.95,0.92}

\lstdefinestyle{mystyle}
{
    backgroundcolor=\color{backcolour},   
    commentstyle=\color{codegreen},
    keywordstyle=\color{magenta},
    numberstyle=\tiny\color{codegray},
    stringstyle=\color{codepurple},
    basicstyle=\ttfamily\footnotesize,
    breakatwhitespace=false,         
    breaklines=true,                 
    captionpos=b,                    
    keepspaces=true,                 
    numbers=left,                    
    numbersep=5pt,                  
    showspaces=false,                
    showstringspaces=false,
    showtabs=false,                  
    tabsize=2
}

\lstset{style=mystyle}


%\makeindex

\begin{document}

\title{%
 Zadanie Programistyczne 2:\\
\large Cykle}
\author{Maksymilian Neumann}

\maketitle

\fancyhf{}
\renewcommand{\headrulewidth}{2pt}
\renewcommand{\footrulewidth}{2pt}
\fancyhead{\leftmark}
\fancyfoot{\thepage}

\newpage


\section{Implementacja Fisher-Yates shuffles}

Funkcja przyjmuje $n \in N^+$ i zwraca jej losową permutacje należącą do $S_n$.\\

\begin{lstlisting}[language=Python, caption=Implementacja Fisher-Yates shuffles]
import random

def shuffle(n):
    base = list(range(1, n+1))#element neutralny permutacji
    permutation = []
    for i in range(len(base)):
        temp = random.randint(0, (len(base)-1))
        permutation.insert(0, base[temp])
        base.pop(temp)
    return permutation
\end{lstlisting}

\section{Rozkład Permutacji na Cykle}

Funkcja przyjmuje permutacje w postaci listy i zwraca rozkład jej na cykle w postaci listy list.\\

\begin{lstlisting}[language=Python, caption=Rozkład Permutacji na cykle]
def cycler(permutation):
    cycles = []#lista cykli
    for i in range(1, len(permutation) + 1):
        if permutation[i - 1] != 0: #dodaje cykl do listy
            temp = i
            cycle = []
            cycle.append(i)
            while permutation[temp - 1] != i:
                cycle.append(permutation[temp - 1])
                temp = permutation[temp - 1]
            cycles.append(cycle)
            for j in cycle: #oznacza jako uzyte w cyklu
                permutation[permutation.index(j)] = 0
    return cycles
\end{lstlisting}

\newpage

\section{Test Działania Kodu}

Z wykładu dowiedzieliśmy się ,że dla $\pi \in S_{n}$ ,gdzie $L(\pi)$ = liczba cykli permutacji $\pi$ \[ E(L) = H_{n}\]
która będzie naszą oczekiwaną wartością\\

\begin{lstlisting}[language=Python, caption=Test zapisyjący wyniki do pliku]
from statistics import mean

def test():
    f = open("test.csv", 'w')#plik z wynikami
    f.write("n;avrage number of cycles\n")
    for n in range(1, 101):
        lengths = []#dlugosci cykli
        for i in range(8000):
            lengths.append(len(cycler(shuffle(n))))
        row = str(n) + ";" + str(mean(lengths)) + "\n"
        f.write(row)#wpisuje n i srednia dlugosc
    f.close()
\end{lstlisting}

\begin{center}
{\Large \textbf{Wyniki:}}
\end{center}

\begin{table}[hb]
\centering
\caption{Wyniki eksperymentu}
\label{tab:my-table}
\begin{tabular}{|l|l|l|l|l|l|l|}
\hline
\textbf{n} & \textbf{wynik} & \textbf{n} & \textbf{wynik} & ... & \textbf{n} & \textbf{wynik} \\ \hline
1          & 1              & 11         & 3.0295         & ... & 91         & 5.10725        \\ \hline
2          & 1.49675        & 12         & 3.104375       & ... & 92         & 5.07775        \\ \hline
3          & 1.84175        & 13         & 3.18375        & ... & 93         & 5.095875       \\ \hline
4          & 2.086625       & 14         & 3.269375       & ... & 94         & 5.139875       \\ \hline
5          & 2.294625       & 15         & 3.32475        & ... & 95         & 5.099875       \\ \hline
6          & 2.46375        & 16         & 3.395125       & ... & 96         & 5.17275        \\ \hline
7          & 2.5915         & 17         & 3.45175        & ... & 97         & 5.17175        \\ \hline
8          & 2.7125         & 18         & 3.49375        & ... & 98         & 5.16425        \\ \hline
9          & 2.8165         & 19         & 3.520625       & ... & 99         & 5.164375       \\ \hline
10         & 2.915125       & 20         & 3.615375       & ... & 100        & 5.191          \\ \hline
\end{tabular}
\end{table}

\begin{figure}[]
 \caption{Lewy wykres przedstawia wyniki eksperymentu, Prawy wartości oczekiwane}
\includegraphics[width=7.5cm]{Test.png}
\includegraphics[width=7.5cm]{Harmonic.png}

\end{figure}

\begin{figure}[]
\centering
 \caption{Wykres wyników eksperymentu nałożony na wartości oczekiwane}
\includegraphics[width=15cm]{Overlayed.png}

\end{figure}
\newpage
Wyniki eksperymentu sugerują poprawność kodu.

\section{Główne Zadanie}

\begin{center}
\textbf{\Large{Hipoteza:}}
\end{center}
Można zauważyć, że dla $\pi \in S_{n}$ ,gdzie $\pi$ = $C_1 \circ$…$\circ C_k$ 
\begin{center}
1. $M_n$ = max(len($C_1$),…,len($C_k$)) = $\lceil \frac{n}{k} \rceil$ 
\end{center}
2.Wiemy również że takich maksymalnych długości jest $\stirlingone{n}{k}$\\
\begin{center}
Z 1. i 2. wynika:
\end{center}
\[E(M_n) = \frac{\sum_{k=1}^{n} \lceil \frac{n}{k} \rceil \stirlingone{n}{k}}{n!}\]
\newpage

\begin{center}
\textbf{\Large{Eksperyment:}}
\end{center}

\begin{lstlisting}[language=Python, caption=Eksperyment numeryczny dla głównego zadania]
from statistics import mean

def zadanie():
    f = open("zadanie.csv", 'w')#plik z wynikami
    f.write("n;avrage max length of a cycle in a permutation\n")
    for n in range(1, 101):
        maxLengths = []#maksymalne dlugosci
        for i in range(8000):
            maxLengths.append(max(len(c) for c in cycler(shuffle(n))))
        row = str(n) + ";" + str(mean(maxLengths)) + "\n"
        f.write(row)#wpisuje n i srednia max dlugosc
    f.close()
\end{lstlisting}

\begin{center}
\textbf{\Large{Wyniki:}}
\end{center}
\begin{table}[hb]
\centering
\caption{Wyniki eksperymentu}
\label{tab:my-table}
\begin{tabular}{|l|l|l|l|l|l|l|}
\hline
\textbf{n} & \textbf{wynik} & \textbf{n} & \textbf{wynik} & ... & \textbf{n} & \textbf{wynik} \\ \hline
1          & 1              & 11         & 7.2196         & ... & 91         & 56.21          \\ \hline
2          & 1.5024         & 12         & 7.8068         & ... & 92         & 57.1102        \\ \hline
3          & 2.167          & 13         & 8.4079         & ... & 93         & 58.2188        \\ \hline
4          & 2.7766         & 14         & 9.0534         & ... & 94         & 59.1646        \\ \hline
5          & 3.4133         & 15         & 9.64           & ... & 95         & 59.7959        \\ \hline
6          & 4.0509         & 16         & 10.3097        & ... & 96         & 60.2964        \\ \hline
7          & 4.6599         & 17         & 10.8826        & ... & 97         & 60.9523        \\ \hline
8          & 5.2903         & 18         & 11.5868        & ... & 98         & 61.4482        \\ \hline
9          & 5.8982         & 19         & 12.1614        & ... & 99         & 62.2856        \\ \hline
10         & 6.5387         & 20         & 12.7366        & ... & 100        & 62.3016        \\ \hline
\end{tabular}
\end{table}

\begin{figure}[]
\centering
 \caption{Wykres wyników eksperymentu}
\includegraphics[width=15cm]{Experiment.png}

\end{figure}
\newpage

\begin{center}
\textbf{\Large{Wnioski:}}
\end{center}
Hipoteza ta okazała się być płytka, naiwna i błędna. Poprawna jedynie dla $n \leq 3$ . Prawdziwa Odpowiedź okazała się być dość skomplikowana i fascynująca o, której dowiedziałem się dzięki "Probabilistic Methods in Discrete Mathematics" 

\newpage

\section{Referencje}

\begin{itemize}

\item Wykresy stworzone przy użyciu biblioteki seaborn
\item \href{https://github.com/Spiryd/cykle}{Repozytorium z zadaniem}

\end{itemize}

\end{document}
