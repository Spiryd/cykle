\documentclass[a4paper,12pt]{article}

\usepackage{polski}
\usepackage[utf8]{inputenc}

\usepackage[english]{babel}
\usepackage{blindtext}
\usepackage{microtype}
\usepackage{graphicx}
\usepackage{fancyhdr}
\usepackage{amsmath}
\usepackage{index}
\usepackage{textgreek}

\usepackage{listings}
\usepackage{xcolor}


\newcommand{\stirlingone}[2]{\genfrac{[}{]}{0pt}{}{#1}{#2}}

\definecolor{codegreen}{rgb}{0,0.6,0}
\definecolor{codegray}{rgb}{0.5,0.5,0.5}
\definecolor{codepurple}{rgb}{0.58,0,0.82}
\definecolor{backcolour}{rgb}{0.95,0.95,0.92}

\lstdefinestyle{mystyle}
{
    backgroundcolor=\color{backcolour},   
    commentstyle=\color{codegreen},
    keywordstyle=\color{magenta},
    numberstyle=\tiny\color{codegray},
    stringstyle=\color{codepurple},
    basicstyle=\ttfamily\footnotesize,
    breakatwhitespace=false,         
    breaklines=true,                 
    captionpos=b,                    
    keepspaces=true,                 
    numbers=left,                    
    numbersep=5pt,                  
    showspaces=false,                
    showstringspaces=false,
    showtabs=false,                  
    tabsize=2
}

\lstset{style=mystyle}


%\makeindex

\begin{document}

\title{%
 Zadanie Programistyczne 2:\\
\large Cykle}
\author{Maksymilian Neumann}

\maketitle

\fancyhf{}
\renewcommand{\headrulewidth}{2pt}
\renewcommand{\footrulewidth}{2pt}
\fancyhead{\leftmark}
\fancyfoot{\thepage}

\newpage


\section{Implementacja Fisher-Yates shuffles}

Funkcja przyjmuje listę w postaci [1,2,…,n](element neutralny) i zwraca jej losową permutacje.\\

\begin{lstlisting}[language=Python, caption=Implementacja Fisher-Yates shuffles]
import random

def shuffle(given):
    permutation = []
    for i in range(len(given)):
        temp = random.randint(0, (len(given)-1))
        permutation.insert(0, given[temp])
        given.pop(temp)
    return permutation
\end{lstlisting}

\section{Rozkład Permutacji na Cykle}

Funkcja przyjmuje permutacje w postaci listy i zwraca rozkład jej na cykle w postaci listy list.\\

\begin{lstlisting}[language=Python, caption=Rozkład Permutacji na cykle]
def cycler(permutation):
    cycles = []
    for i in range(1, len(permutation) + 1):
        if permutation[i - 1] != 0: 
            temp = i
            cycle = []
            cycle.append(i)
            while permutation[temp - 1] != i:
                cycle.append(permutation[temp - 1])
                temp = permutation[temp - 1]
            cycles.append(cycle)
            for j in cycle: 
                permutation[permutation.index(j)] = 0
    return cycles
\end{lstlisting}

\newpage

\section{Test Działania Kodu}

Z wykładu dowiedzieliśmy się ,że dla $\pi \in S_{n}$ ,gdzie $L(\pi)$ = liczba cykli permutacji $\pi$ \[ E(L) = H_{n}\]
która będzie naszą oczekiwaną wartością\\

\begin{lstlisting}[language=Python, caption=Test zapisyjący wyniki do pliku]
from statistics import mean

def test():
    f = open("test.csv", 'w')
    f.write("n;avrage number of cycles cycles\n")
    for n in range(1, 101):
        lengths = []
        for i in range(8000):
            x = list(range(1, n + 1))
            lengths.append(len(cycler(shuffle(x))))
        row = str(n) + ";" + str(mean(lengths)) + "\n"
        f.write(row)
    f.close()
\end{lstlisting}

\begin{figure}[hb]
 \caption{Lewy wykres przedstawia wyniki eksperymentu, Prawy wartości oczekiwane}
\includegraphics[width=7.5cm]{Test.png}
\includegraphics[width=7.5cm]{Harmonic.png}

\end{figure}

\begin{figure}[ht]
\centering
 \caption{Wykres wyników eksperymentu nałożony na wartości oczekiwane}
\includegraphics[width=15cm]{Overlayed.png}

\end{figure}
\newpage
Wyniki eksperymentu sugerują poprawność kodu.

\section{Główne Zadanie}

\begin{center}
\textbf{\Large{Hipoteza:}}
\end{center}
Można zauważyć, że dla $\pi \in S_{n}$ ,gdzie $\pi$ = $C_1 \circ$…$\circ C_k$ 
\begin{center}
1. $M_n$ = max(len($C_1$),…,len($C_k$)) = $\lceil \frac{n}{k} \rceil$ 
\end{center}
2.Wiemy również że takich maksymalnych długości jest $\stirlingone{n}{k}$\\
\begin{center}
Z 1. i 2. wynika:
\end{center}
\[E(M_n) = \frac{\sum_{k=1}^{n} \lceil \frac{n}{k} \rceil \stirlingone{n}{k}}{n!}\]



\begin{center}
\textbf{\Large{Eksperyment:}}
\end{center}

\begin{lstlisting}[language=Python, caption=Eksperyment numeryczny dla głównego zadania]
from statistics import mean

def zadanie():
    f = open("zadanie.csv", 'w')
    f.write("n;avrage max length of a cycle in a permutation\n")
    for n in range(1, 101):
        maxLengths = []
        for i in range(8000):
            x = list(range(1, n + 1))
            maxLengths.append(max(len(c) for c in cycler(shuffle(x))))
        row = str(n) + ";" + str(mean(maxLengths)) + "\n"
        f.write(row)    
    f.close()
\end{lstlisting}


\end{document}
